
% Comments

\documentclass{irodsugm}
\usepackage{float}
\usepackage{graphicx}
\usepackage{url}
\usepackage{tikz}
\usepackage{csquotes}

\begin{document}

% --- Copyright Notice here ---
\conferenceinfo{iRODS UGM 2021}{June 8-11, 2021, Virtual\\
{[}Authors retain copyright.{]}}
% --- End of Copyright Notice ---

\title{iRODS Logical Locking}
%
% You need the command \numberofauthors to handle the 'placement
% and alignment' of the authors beneath the title.
%
% For aesthetic reasons, we recommend 'three authors at a time'
% i.e. three 'name/affiliation blocks' be placed beneath the title.
%
% NOTE: You are NOT restricted in how many 'rows' of
% "name/affiliations" may appear. We just ask that you restrict
% the number of 'columns' to three.
%
% Because of the available 'opening page real-estate'
% we ask you to refrain from putting more than six authors
% (two rows with three columns) beneath the article title.
% More than six makes the first-page appear very cluttered indeed.
%
% Use the \alignauthor commands to handle the names
% and affiliations for an 'aesthetic maximum' of six authors.
% Add names, affiliations, addresses for
% the seventh etc. author(s) as the argument for the
% \additionalauthors command.
% These 'additional authors' will be output/set for you
% without further effort on your part as the last section in
% the body of your article BEFORE References or any Appendices.

\numberofauthors{4} %  in this sample file, there are a *total*
% of THREE authors. If greater than six authors are specified,
% SIX will appear on the 'first-page' (for formatting
% reasons) and the remaining two will appear in the \additionalauthors section.
%
\author{
% You can go ahead and credit any number of authors here,
% e.g. one 'row of three' or two rows (consisting of one row of three
% and a second row of one, two or three).
%
% The command \alignauthor (no curly braces needed) should
% precede each author name, affiliation/snail-mail address and
% e-mail address. Additionally, tag each line of
% affiliation/address with \affaddr, and tag the
% e-mail address with \email.
%
% 1st. author
\alignauthor
Alan King \\
       \affaddr{Renaissance Computing Institute (RENCI)} \\
       \affaddr{UNC Chapel Hill}
       \email{alanking@renci.org}

\alignauthor
Terrell Russell \\
       \affaddr{Renaissance Computing Institute (RENCI)} \\
       \affaddr{UNC Chapel Hill}
       \email{unc@terrellrussell.com}
}
% There's nothing stopping you putting the seventh, eighth, etc.
% author on the opening page (as the 'third row') but we ask,
% for aesthetic reasons that you place these 'additional authors'
% in the \additional authors block, viz.
%\additionalauthors{Additional authors: Seventh Author (Affiliation, 
% email: {\texttt{email@addre.ss}}) and Eighth Author (Affiliation, email: {\texttt{email@addre.ss}}).}

% Just remember to make sure that the TOTAL number of authors
% is the number that will appear on the first page PLUS the
% number that will appear in the \additionalauthors section.

\date{11 June 2021}

\maketitle

\begin{abstract}
iRODS 4.2.9 introduces Logical Locking by providing additional replica status values within the catalog. Previously, replicas in iRODS could only be marked 'good' or 'stale'. This did not capture the states of when data was in flight, or incomplete. This talk will explain the new intermediate and locked states for iRODS replicas and how they are used to provide protection from uncoordinated writes into the system.
\end{abstract}

\keywords{iRODS, data management, logical locking, concurrency}


\section*{INTRODUCTION}

The iRODS Technical Overview\cite{technicaloverview} states:

\begin{displayquote}
iRODS provides a logical representation of files stored in physical storage locations. We call this logical view a virtual file system and the capabilities it provides, Data Virtualization.
\end{displayquote}

Interactions with data in an iRODS system are meant to mirror how one might interact with data in a traditional hierarchical file system. The purpose of the virtual file system is to present data across a number of geographically distributed servers under a unified namespace - in other words, a distributed system. A distributed system implies operational concurrency and operational concurrency implies race conditions.

The following document will describe how iRODS interacts with data, how the iRODS Consortium seeks to address concurrency in manipulation of data within an iRODS system, and planned efforts for protecting data access and manipulation as well as policy execution.

\section*{INTERACTING WITH DATA IN iRODS}

\subsection*{History of data representation and movement}

The logical representation of data presented by the virtual file system is known as a Data Object, which the iRODS Beginner Training\cite{beginnertraining} defines as:

\begin{displayquote}
\textbf{Data Object}: a logical representation of data that maps to one or more physical instances (\textbf{Replicas}) of the data at rest in Storage Resources
\end{displayquote}

A Replica could be one of two statuses:

\begin{itemize}
\item \textbf{Stale}: The replica is no longer known to be good. Usually indicates that a more recently written-to replica exists. This does not necessarily mean that the data is incorrect but merely that it may not reflect the "truth" of this data object.
\item \textbf{Good}: The replica is good. Usually the latest replica to be written. When a replica is good, all bytes and system metadata (size and checksum, if present) are understood to have been recorded correctly.
\end{itemize}

iRODS has historically presented an interface which interacts with data at the logical level - that is, with data objects. However, over time, it became clear that interacting directly with replicas is very important for iRODS users and administrators. Other operations were added to allow for replica manipulation which were very difficult to integrate with existing APIs.

As such, operations which deal directly with replicas have completely separate implementations for moving data from the logical level operations. Furthermore, operations dealing with data objects still need access to replica information (after all, the operations interact with the storage eventually, at which point it is interacting at the replica level). 

On top of all of this, because replica status can only be good or stale, any replica which exists in the catalog but has not been finalized is being represented incorrectly because the data is still in flight. Good and stale refer to data which is at rest. There is no status to represent in-flight data.

\subsection*{Unifying data movement with replicas as First Class Citizens}

These differences in implementation have consistency and performance implications for moving data. In reality, all of these operations should be and are identical:

\begin{samepage}
\begin{center}
%\begin{verbatim}
Open replica -> move data to replica -> close replica -> finalize system metadata in catalog
%\end{verbatim}
\end{center}
\end{samepage}

Every data moving operation in iRODS should implement the above steps. These should use the same set of low-level interfaces so that data movement is made consistent across operations. To make this possible, replicas must be treated as the operative entity in data movement, not data objects (which are one-to-many mappings to replicas). This requires that replicas be treated as First Class Citizens within the iRODS server where, historically, they have not - at least not in all instances.

Efforts are being made to change this in the server, but it has not been fully realized. We will take a slight detour to describe in detail one of the most important operations within an iRODS system: replication. First it will be defined at an abstract level as it is intended to function, and then the implementation and interface in iRODS will be described. This highlights the importance of a unified mode of data movement within the system even while the requirements for the operation differ so greatly from others such as put, copy, and rename.

\section*{RULES FOR REPLICATION}

\textbf{Replication} refers to copying a physical replica of an existing data object from one storage resource to another. Replication requires 3 inputs:
\begin{enumerate}
    \item \textbf{Logical path}: Path to an existing data object with at least one replica which is at rest and can be read.
    \item \textbf{Source resource}: The resource from which data will be copied/read.
    \item \textbf{Destination resource}: The resource to which data will be copied/written.
\end{enumerate}

The source resource must already have an existing, at-rest replica which can be read in order for replication to be \textit{possible} in any case.

If the destination resource has no replica, replication to the destination resource is \textit{allowed} in \textbf{all cases}. If this is true, a \textbf{new replica} will appear on the destination resource as a result of the replication. If this is not true, the destination resource has an existing replica and replication may or may not occur.

If the destination resource has an existing replica, replication would be performing a \textbf{replica update}. The following requirements must be true in order for updating the replica to be \textit{allowed}:
\begin{enumerate}
    \item The destination resource must not be the source resource
    \item The destination replica must be stale
    \item The source replica must be good
\end{enumerate}

\begin{table}[]
    \centering
    \begin{tabular}{c | c | c | c | c}
        case & source & destination & result & reason \\
        \hline
        0 & - & - & unchanged & No source replica exists \\
        1 & - & \& & unchanged & No source replica exists \\
        2 & - & X & unchanged & No source replica exists \\
        3 & \& & - & \& & Replication allowed \\
        4 & \& & \& & unchanged & Destination replica must be stale \\
        5 & \& & X & \& & Replication allowed \\
        6 & X & - & X & Replication allowed \\
        7 & X & \& & unchanged & Destination replica must be stale \\
        8 & X & X & unchanged & Source replica must be good
    \end{tabular}
    \caption{This table shows the initial state of the replica on the source and destination resources, the resulting replica state on the destination resource, and the reason for the result.}
    \label{tab:replication_behavior_table}
\end{table}

\textbf{Table 1} shows all possible cases for replication between a single source and destination when the source and destination replicas do or do not exist. Some notes about the table:
\begin{itemize}
    \item ‘X’ represents a stale replica
    \item ‘\&’ represents a good replica
    \item ‘-’ indicates a replica which does not exist.
    \item All replicas are assumed to be at rest
\end{itemize}

\section*{iRODS IMPLEMENTATION OF REPLICATION}

The iRODS server provides the \texttt{rsDataObjRepl} API in order to replicate data objects. Here is the signature:

\begin{samepage}
\begin{verbatim}
    int rsDataObjRepl(rsComm_t*, dataObjInp_t*, transferStat_t**);
\end{verbatim}
\end{samepage}

As described in the Rules for Replication, replication requires a logical path, a source resource, and a destination resource. The following describes how these requirements can be satisfied.

\subsection*{Logical Path}
As described in the general Rules for Replication section, the logical path must refer to an existing iRODS data object of which the to-be-determined-or-specified source replica must exist. The authenticated user must have at least write permssion on the data object in order to read from the source replica and write to the destination replica. The other qualifications regarding source and destination replica statuses have already been described in the Rules for Replication.

\subsection*{Source Resource}
In order to specify a source resource, any of the following may be provided via \texttt{condInput}:

\begin{enumerate}
    \item \texttt{RESC\_NAME\_KW} - Must be a root resource. If this is not true, the following error is returned: \texttt{DIRECT\_CHILD\_ACCESS}.
    \item \texttt{RESC\_HIER\_STR\_KW} - Must be a full resource hierarchy. No resource hierarchy resolution is performed.
    \item \texttt{REPL\_NUM\_KW}
\end{enumerate}

If none of the above is provided, any resource is eligible for use as the source resource.

Resource hierarchy resolution is then performed on a \texttt{OPEN} operation by the API to determine the full resource hierarchy. The following must be true of the resolve resource hierarchy (depending on which of the destination resource inputs was provided):

\begin{enumerate}
    \item The resolved resource hierarchy has a root which matches \texttt{RESC\_NAME\_KW}
    \item No resource hierarchy resolution is performed.
    \item The resolved resource hierarchy has an existing replica with a replica number which matches \texttt{REPL\_NUM\_KW}
\end{enumerate}

If the resolved resource hierarchy does not meet the requirements of the provided inputs or the requirements of the source replica as defined by the rules of replication, the following error is returned: \texttt{SYS\_REPLICA\_INACCESSIBLE}.

\subsection*{Destination resource}
In order to specify a destination resource, any of the following may be provided via \texttt{condInput}:

\begin{enumerate}
    \item \texttt{DEST\_RESC\_NAME\_KW} - Must be a root resource. If this is not true, the following error is returned: \texttt{DIRECT\_CHILD\_ACCESS}.
    \item \texttt{DEST\_RESC\_HIER\_STR\_KW} - Must be a full resource hierarchy.
\end{enumerate}

If neither of the above is provided, the system will look for a \texttt{DEF\_RESC\_NAME\_KW}. If present, \texttt{DEST\_RESC\_NAME\_KW} is set to this value. If not, any resource is eligible for use as the destination resource.

Resource hierarchy resolution is then performed on a \texttt{CREATE} operation by the API to determine the full resource hierarchy. The following must be true of the resolve resource hierarchy (depending on which of the destination resource inputs was provided):

\begin{enumerate}
    \item The resolved resource hierarchy has a root which matches \texttt{DEST\_RESC\_NAME\_KW}
    \item No resource hierarchy resolution is performed.
\end{enumerate}

If the resolved resource hierarchy does not meet the requirements of the provided inputs, the following error is returned: \texttt{SYS\_REPLICA\_INACCESSIBLE}.

If the destination resource has a replica which does not meet the requirements for a destination replica as defined in the rules of replication, the following error will be returned: \texttt{SYS\_NOT\_ALLOWED}.

\subsection*{Updating all replicas}
When the \texttt{ALL\_KW} is provided, the API will gather a list of all existing replicas and update each one according to all of the rules described above. Specifying a destination resource along with the \texttt{ALL\_KW} is not allowed.

\subsection*{\texttt{irepl}}
The historic iCommand provided for replicating data objects from the command line is \texttt{irepl}. 

\section*{UNCOORDINATED, CONCURRENT OPERATIONS ON SYSTEM STATE/DATA}

Once data movement is unified across the iRODS server, the conversation about concurrency as it relates to system correctness can happen. We have identified three ways in which the state of the system can be made incorrect:

\begin{enumerate}
\item Uncoordinated, concurrent writing to a single replica can lead to \textbf{Data Corruption}.
\item Uncoordinated, concurrent writing to multiple replicas of the same data object can lead to \textbf{Truth Corruption}.
\item Uncoordinated, concurrent operation execution can lead to \textbf{Policy Violations}.
\end{enumerate}

The iRODS Consortium is actively working to address these three situations in the server. The first two have been partially addressed in released server software and the third is to be addressed in a future release.

\subsection*{Data Corruption and Intermediate Replicas}

The first problem listed above is that in-flight replicas can be opened and modified concurrently by multiple agents in an uncoordinated fashion. Also, as a result of iRODS historically creating replicas in the Good status, the catalog does not reflect the current, true state of the data from the moment it is registered in the catalog.

To address this, a third replica status has been introduced:

\textbf{Intermediate:} The replica is actively being written to. The state of the replica cannot be determined because the client writing to it still has the replica opened. Replicas which are marked intermediate cannot be opened for read or write, nor can they be unlinked or renamed.

Replicas are created in the Intermediate status for data moving operations because the replica is considered in-flight until the time that it is finalized in the catalog. In this way, the replica status is accurately represented in the catalog and the system is able to respond appropriately to concurrent accesses to this replica.

\subsection*{Truth Corruption and Logical Locking}

However, it is unclear which replica for a given data object represents the Truth when multiple replicas are in flight at the same time. Protecting individual replicas is not enough to verify the correctness of the data at the logical level.

To address this, a new mechanism called iRODS Logical Locking (ILL) has been introduced which disallows opening any replica for a given data object when any one of the replicas are opened for write. This has been implemented using another new (fourth) replica status:

\textbf{Write-Locked:} One of this replica's sibling replicas is actively being written to but is itself at rest. Replicas which are write-locked cannot be opened for read or write, nor can they be unlinked or renamed.

By only allowing coordinated modifications to a data object, assertions about the correctness of the logical data object can be made based upon the correctness of a replica because multiple replicas cannot simultaneously claim to be correct (while potentially being different).

\section*{DESIGN}

iRODS Logical Locking operates on the basic principles of a traditional read-write lock in computing\cite{readerswriterlock}

\begin{enumerate}
\item The data object is locked when any of its replicas are opened.
\item The data object is unlocked when the opened replica is closed/finalized.
\end{enumerate}

Here, we describe this process in more detail:

\subsection*{The data object is locked when any of its replicas are opened}

\begin{enumerate}
\item Client requests to open a replica of a given data object.
\item iRODS checks the status of the replica to make sure it is not already locked (equivalent to trying to "acquire" the lock). "Locked" in this case means:
   \begin{enumerate}
   \item the replica is read-locked (this or a sibling replica is opened for read by an agent): the operation will fail if it is not an open-for-read.
   \item the replica is write-locked (a sibling replica is intermediate): the operation will fail.
   \item the replica is intermediate: the operation will fail for an uncoordinated open-for-write or any open-for-read. See the Implementation Details section for details about concurrent write coordination.
   \item If the open intends to create a new replica, the operation will fail if any replica is found not to be at rest (that is, intermediate, read-locked, or write-locked).
   \end{enumerate}
\item Depending on the operation, the data object is then locked for this open. "Locked" in this case means:
   \begin{enumerate}
   \item \textbf{open-for-read}: the status of all replicas is set to read-lock
   \item \textbf{open-for-write}: the status of the target replica is set to intermediate and the statuses of the sibling replicas are set to write-lock
   \item \textbf{open-for-create}: the statuses of any existing sibling replicas are set to write-locked
   \end{enumerate}
\item The target replica is then physically opened and the open is complete. If a new replica is supposed to have been created, the entry is created in the catalog in the intermediate status.
\end{enumerate}

As we will see later, there is still a race condition here: the TOCTOU\cite{toctou} condition in the iRODS catalog itself. We have designed a solution for this which will be described later, but we mention it here to acknowledge the situation which undoubtedly entered the minds of some readers.

\subsection*{The data object is unlocked when the opened replica is closed/finalized}

\begin{enumerate}
\item The client requests to close and finalize a replica of a given data object.
\item The data object is unlocked atomically along with updating the other system metadata. The final states of the replicas depends mainly on the operation requesting the close. The general cases for unlocking are shown here, but some operations (e.g. replication, phymv, etc.) may behave differently as they have different requirements:
   \begin{enumerate}
   \item \textbf{open-for-read}: if no other agents have any of the replicas opened, the replica statuses of each replica is restored to what it was before the open
   \item \textbf{open-for-write}: there are two cases:
      \begin{enumerate}
      \item success: the target replica is marked good and all sibling replicas are marked stale
      \item failure: the target replica is marked stale and all sibling replicas have their statuses restored to what they were before the open
      \end{enumerate}
   \end{enumerate}
\end{enumerate}

\section*{IMPLEMENTATION DETAILS (IN 4.2.9+)}
\subsection*{Code Facilities}
Two mechanisms were developed to implement logical locking:

\begin{enumerate}
\item \texttt{data\_object\_finalize} API plugin (finalize) - atomically applies updates to rows in \texttt{R\_DATA\_MAIN}
\item \texttt{replica\_state\_table} (RST) - An in-memory, per-agent JSON structure which describes a snapshot of rows in \texttt{R\_DATA\_MAIN} for a particular data\_id
\end{enumerate}

A library was developed under the \texttt{irods::logical\_locking} namespace in the server on top of these utilities which makes surgical edits to the RST and updates the catalog via the finalize API to atomically update - and therefore unlock - data objects. As described in the design, data objects are locked on open and unlocked on close/finalize using this \texttt{logical\_locking} library.

\subsection*{Refactoring iRODS Internals}
Much refactoring was necessary in the iRODS server for a correct first attempt at an ILL implementation and to fix critical existing issues, all while maintaining the existing client-facing interfaces. In order to ensure that ILL is enforced across the API surface, the requirements described in "Unifying data movement with replicas as First Class Citizens" needed to be put in place. This accomplishes both the aforementioned consistency and correctness of data movement throughout the system, and a single entry point for using ILL.

A major shift was made internally in the iRODS server away from relying on \texttt{rsDataObjClose} to handle the logic of finalizing replicas as has been done historically. Each operation has its own requirements for how the replicas in a data object should appear for success or failure cases and in many cases they differ from those of other operations. The finalize API plugin is now used in conjunction with the \texttt{replica\_close} API plugin to determine and control how replicas should be updated after the completion of an operation. \texttt{rsDataObjClose} can still be used to close and finalize a data object, but the resultant replica statuses are handled generically.

\subsection*{Coordinated, concurrent write support}
To support concurrent writes to the same replica which enables parallel data transfer using multiple iRODS agents, another mechanism was developed called the Replica Access Registry (RAR). The RAR is held in shared memory so that agents spawned from a particular iRODS server can coordinate concurrent writes to particular replicas. When an iRODS agent opens a replica for write (or create) an entry is created in the RAR which associates an agent PID with a replica of a data object and is accessed via a universally unique identifier called a Replica Access Token (RAT). This token is part of the L1 descriptor held in the agent process which initiated the replica open. This token allows other iRODS agents connected to the same server to open the same intermediate replica.

Using these tokens, clients can now implement parallel data transfer to a single replica over the main iRODS port. This has been demonstrated both in the C++ client API with the Parallel Transfer Engine (PTE) TODO CITE and in the Python iRODS Client (PRC)\cite{prc} and is included in the 1.0.0 release.

\section*{KNOWN LIMITATIONS}

As with most features, ILL does not protect against rogue administrators.

\subsection*{Performance}
Now that the server is enforcing coordination of writes across the iRODS Zone, this necessarily means more trips to the iRODS Catalog to query for information and modify rows to update replica statuses.

\subsection*{The Database Race}
ILL is currently still prone to database race conditions in the iRODS Catalog due to a classic TOCTOU\cite{toctou} problem which allows multiple winners for the race. The scenario arises when two agents arrive to lock a data object at the same time. The lock is first checked and then set if it passes. All at-rest, unopened data objects are eligible to be locked. The two agents will both find that the object is unlocked and then go to the database to set the status. The database will not stop this and both agents will think that they have exclusive access to their respectively opened replicas of the data object.

We think that this can be solved by causing the check and set of the replica statuses to be a single database transaction. In this way, the system will lean upon the locking mechanism of the database and the lock cannot be "acquired" by more than one agent and so there is only one winner.

\section*{FUTURE WORK}

\subsection*{Policy Violation (Operation Locking)}

The third problem listed before is violations in configured policy due to uncoordinated, concurrent policy invocations which incur data-modifying operations. We think that this will be solved by operation-aware logical locking, or \textbf{Operation Locking}. Where ILL is currently gated on opening and closing replicas of a data object, Operation Locking would be gated on the start and completion of a high-level operation.

As described before, open and close are considered "low-level" interfaces where things like put, replication, and phymv are "high-level" and are built upon the low-level interfaces. Operation Locking would move the locking mechanism up from open into the higher level interfaces so that locking can be controlled across multiple data-modifying operations resulting from policy invocation. In this way, policy invocation will be made more correct and even enable coordinated, concurrent data movement within the policy-triggered operation(s). This will require extending ILL to make iRODS agents more aware of which lock(s) they have acquired.

\subsection*{Read Locks}

The goal behind ILL is to implement a readers-writer lock\cite{readerswriterlock} on data object resources via the replica status in the iRODS Catalog. A read lock will allow for multiple, \textbf{uncoordinated} agents to read from any replica in a data object at the same time. When a data object is read-locked, all data-modifying operations should be blocked (open for write, unlink, rename, etc.) Unlike write locks, which require coordination to lock and unlock properly, read locks will allow for uncoordinated reads. This means that ILL will need to be extended to track a list of agent PIDs and hostnames which are holding the lock. This may imply a refactor of the current lock implementation so that the solution can remain generic and possibly bring it more in line semantically with traditional locks/mutexes.

\section*{SUMMARY}

TODO big finish

\begin{thebibliography}{5}

\bibitem{technicaloverview}iRODS Technical Overview (2016). \url{https://irods.org/uploads/2016/06/technical-overview-2016-web.pdf}

\bibitem{beginnertraining}iRODS Beginner Training (2019). \url{https://github.com/irods/irods_training/blob/5418c42d7d6c33d92b14acc02f9bbba5c414b779/beginner/irods_beginner_training_2019.pdf}

\bibitem{readerswriterlock}Lamport, Leslie; \textit{Concurrent Reading and Writing}; Communications of the ACM, November 1977, Vol. 20 No. 11, Pages 806-811 \url{http://lamport.azurewebsites.net/pubs/rd-wr.pdf}

\bibitem{toctou}Time of Check Time of Use \url{https://cwe.mitre.org/data/definitions/367.html}

\bibitem{pte}Parallel Transfer Engine \url{need_url}

\bibitem{prc}Python iRODS Client \url{https://github.com/irods/python-irodsclient}


\end{thebibliography}
\end{document}

